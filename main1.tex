\documentclass{ximera}
\usepackage{OERLinearAlgebra}

\usepackage{mathtools}

\author{Paul Zachlin \and Anna Davis} \title{VSP-0050: Abstract Vector Spaces} \license{CC-BY 4.0}
\begin{document}

\begin{abstract}
To understand the properties of abstract vector spaces and to be able to determine if a given set with two operations is a vector space.
\end{abstract}
\maketitle


\section*{VSP-0050: Abstract Vector Spaces}
\subsection*{Properties of Vector Spaces} 

In VSP-M-0020 we discussed $\RR^n$ as a vector space and introduced the notion of a subspace of $\RR^n$.  
%Recall that a subset of of $\RR^n$ is a subspace if the subset is in itself a vector space.  In other words, if it satisfies the following properties...
In this module we will introduces sets other than $\RR^n$ that satisfy the same properties.  Such sets together with the operations of addition and scalar multiplication will also be called vector spaces.

Recall that $\RR^n$, together with operations of vector addition and scalar multiplication satisfies the following properties:
%In the module VEC-M-0030 on Vector Arithmetic, we learned that vectors in $\mathbb{R}^n$ have two operations (addition and scalar multiplication) that satisfy the following eight properties:


  %\begin{theorem}\label{th:vector_properties} The following properties hold for vectors ${\bf u}$, ${\bf v}$ and ${\bf w}$ in $\mathbb{R}^n$ and scalars $k$ and $p$.
  \begin{enumerate}
  \item 
  Commutative Property of Addition  $\quad {\bf u}+{\bf v}={\bf v}+{\bf u}$
  \item 
  Associative Property of Addition $\quad ({\bf u}+{\bf v})+{\bf w}={\bf u}+({\bf v}+{\bf w})$
  \item 
  Existence of Additive Identity  $\quad {\bf u}+{\bf 0}={\bf u}$
  \item 
  Existence of Additive Inverse  $\quad{\bf u}+(-{\bf u})={\bf 0}$
  \item
  Distributive Property over Vector Addition  $\quad k({\bf u}+{\bf v})=k{\bf u}+k{\bf v}$
  \item
  Distributive Property over Scalar Addition  $\quad (k+p){\bf u}=k{\bf u}+p{\bf u}$
  \item 
  Associative Property for Scalar Multiplication $\quad k(p{\bf u})=(kp){\bf u}$
  \item 
  Multiplication by 1 $\quad 1{\bf u}={\bf u}$
  \end{enumerate}
%\end{theorem}


In the module MAT-M-0010 on Matrix Operations and Properties, perhaps you noticed that the set of all $m \times n$ matrices behaves in a similar manner.  That is, we may perform addition and scalar multiplication on matrices, and we find that the same set of properties holds.


  \begin{theorem}\label{matrix_properties} Let ${A}$, ${B}$ and ${C}$ be $m \times n$ matrices and let $k$ and $p$ be scalars.
  
  The following properties hold:
  \begin{enumerate}
  \item 
  Commutative Property of Addition  $\quad A+B=B+A$
  \item 
  Associative Property of Addition $\quad (A+B)+C=A+(B+C)$
  \item 
  Existence of Additive Identity  $\quad A+O=A$ where $O$ is the $m \times n$ zero matrix
  \item 
  Existence of Additive Inverse  $\quad A+(-A)=O$
  \item
  Distributive Property over Matrix Addition  $\quad k(A+B)=kA+kB$
  \item
  Distributive Property over Scalar Addition  $\quad (k+p)A=kA+pA$
  \item 
  Associative Property for Scalar Multiplication $\quad k(pA)=(kp)A$
  \item 
  Multiplication by 1 $\quad 1A=A$
  \end{enumerate}
\end{theorem}


\begin{exploration} Show that the set of all linear functions satisfies each of these eight properties.


 We can write any linear function in slope-intercept form.  Let $f_1(x)=m_1 x + b_1$, $f_2(x)=m_2 x + b_2$, and $f_3(x)=m_3 x + b_3$ be linear functions, and let $k$ and $p$ be scalars.  We check the following:
  \begin{enumerate}
  \item 
  Commutative Property of Addition  
  $$f_1(x) + f_2(x) = (m_1 x + b_1) + (m_2 x + b_2) = (m_2 x + b_2) + (m_1 x + b_1) = f_2(x) + f_1(x) $$
  $$\text{therefore} \quad f_1+f_2=f_2+f_1$$
  \item 
  Associative Property of Addition 
  $$(f_1 + f_2) + f_3 = f_1 + (f_2 + f_3) $$
  \item 
  Existence of Additive Identity  
  $$f_1 + f_0 = f_1 \quad \text{where} \, f_0 \, \text{is the linear function} \, f_0(x)=0.$$
  \item 
  Existence of Additive Inverse  
    $$f_1 + (-f_1) = f_0 \quad \text{where} \, f_0 \, \text{is the linear function} \, f_0(x)=0.$$
  \item
  Distributive Property over Vector Addition 
 $$
  k(f_1(x) + f_2(x)) = k((m_1 x + b_1) + (m_2 x + b_2))  = k(m_1 x + b_1) + k(m_2 x + b_2)  = k f_1(x) + k f_2(x)
  $$
  $$
  \text{therefore}\quad k(f_1+f_2)=kf_1+kf_2
  $$
  \item
  Distributive Property over Scalar Addition  
  $$(k+p)f_1(x)= (k+p)(m_1 x + b_1) =k(m_1 x + b_1) + p(m_1 x + b_1) = k f_1(x) + p f_1(x) $$
  $$\text{therefore}\quad (k+p)f_1=kf_1+pf_1$$
  \item 
  Associative Property for Scalar Multiplication 
  $$k(p(f_1(x)))=k(p(m_1 x + b_1))=k(p m_1 x +p b_1) = (kp) m_1 x + (kp) b_1 = (kp)(m_1 x + b_1)=(kp)f_1(x)$$
  $$\text{therefore}\quad (k(pf_1))=(kp)f_1$$
  \item 
  Multiplication by 1 
  $$1 f_1=f_1$$
  \end{enumerate}
\end{exploration}

The main idea we wish to communicate in this module is that there are many times in mathematics when we encounter a set with two operations (that we call addition and scalar multiplication) that satisfy these same properties.  We recall the concept of a set being \dfn{closed} under an operation that was discussed in module VSP-M-0020.




 

  \begin{definition} 
  A set $V$ is called a \emph{vector space} if it is closed under addition, closed under scalar multiplication, and if the following properties hold for vectors ${\bf u}$, ${\bf v}$ and ${\bf w}$ in $\mathbb{R}^n$ and scalars $k$ and $p$:
  \begin{enumerate}
  \item 
  Commutative Property of Addition  $\quad {\bf u}+{\bf v}={\bf v}+{\bf u}$
  \item 
  Associative Property of Addition $\quad ({\bf u}+{\bf v})+{\bf w}={\bf u}+({\bf v}+{\bf w})$
  \item 
  Existence of Additive Identity  $\quad {\bf u}+{\bf 0}={\bf u}$
  \item 
  Existence of Additive Inverse  $\quad{\bf u}+(-{\bf u})={\bf 0}$
  \item
  Distributive Property over Vector Addition  $\quad k({\bf u}+{\bf v})=k{\bf u}+k{\bf v}$
  \item
  Distributive Property over Scalar Addition  $\quad (k+p){\bf u}=k{\bf u}+p{\bf u}$
  \item 
  Associative Property for Scalar Multiplication $\quad k(p{\bf u})=(kp){\bf u}$
  \item 
  Multiplication by 1 $\quad 1{\bf u}={\bf u}$
  \end{enumerate}
  
\end{definition}


\begin{example}
$\mathbb{R}^n$ is a vector space.  Note that if ${\bf v} \in \mathbb{R}^n$ and ${\bf w} \in \mathbb{R}^n$, then ${\bf v+w} \in \mathbb{R}^n$.  Also, ${\bf v} \in \mathbb{R}^n$ and $c$ is a scalar, then $c{\bf v} \in \mathbb{R}^n$.  This shows that $\mathbb{R}^n$ is closed under both addition and scalar multiplication.  Furthermore, Theorem \ref{vector_properties} states that vectors in $\mathbb{R}^n$ satisfy the other eight properties listed above.  This proves that $\mathbb{R}^n$ is a vector space.
\end{example}

\begin{example}
Let $\mathbb{M}_{m,n}$ be the set of all $m \times n$ matrices.  Then $\mathbb{M}_{m,n}$ is closed under both addition and scalar multiplication.  Also, in Theorem \ref{matrix_properties}, we verified that $\mathbb{M}_{m,n}$ satisfies each of the eight other properties in the definition above.  We conclude that $\mathbb{M}_{m,n}$ is a vector space.  
\end{example}

\begin{example}\label{linear}
Let $F$ be the set of all linear functions.  Then $F$ is closed under both addition and scalar multiplication.  Try to verify this and watch the video that follows Example \ref{deg2} if you need help.  Moreover, in the Initial Problem in this module, we verified that $F$ satisfies each of the eight other properties in the definition above.  We conclude that the set of all linear functions is a vector space.  
\end{example}

\begin{example}\label{deg2}
Let $Y$ be the set of all polynomials in x that are degree two.  To be clear, we define:
$$Y=\left\{ax^2+bx+c : a \ne 0 \right\}$$
We claim that $Y$ is not a vector space, as some of the required properties do not hold.
Observe that $Y$ is not closed under addition.  To see this, let $y_1 = 2x^2+3x+4$ and let $y_2=-2x^2$.  Then $y_1$ and $y_2$ are both elements of $Y$.  However, $y_1+y_2 = 3x+4$ is not an element of $Y$, as it is only a degree one polynomial.  We require the coefficient $a$ of $x^2$ to be nonzero for a polynomial to be in $Y$, and this is not the case for $y_1+y_2$.

As an exercise, check the remaining vector space properties one-by-one to see which properties hold and which do not.  Watch the next video if you need help.
\end{example}

{\color{red}video link}
 \href{https://odu.wistia.com/medias/mtg9f07kyk}

\begin{example}\label{deg_le_2}
So the set $Y$ in Example \ref{deg2} is not a vector space, but we can make a slight modification and we get a vector space.  We define:
$$\mathbb{P}^2=\left\{ax^2+bx+c : a,b,c \in \mathbb{R} \right\}$$
So $\mathbb{P}^2$ consists of polynomials of degree two or less, and note that it includes the "zero polynomial" (when $a=b=c=0$).  We can check that all vector space properties hold, so $\mathbb{P}^2$ is a vector space.  Furthermore, if we define $\mathbb{P}^n$ consists of polynomials of degree $n$ or less, including the "zero polynomial", then by similar reasoning $\mathbb{P}^n$ is a vector space.
\end{example}




\section*{Practice Problems}
\begin{enumerate}

 
\item (Adopted from Kuttler, Exercises 9.1.1-9.1.4) Is the set of all points in $\mathbb{R}^2$ a vector space under the given definitions of addition and scalar multiplication?    In each case be specific about which vector space properties hold and which properties fail.
  \begin{enumerate}
  \item Addition: $(a, b)+(c, d)=(a+d, b+c)$\\ Scalar Multiplication: $k(a, b)=(ka, kb)$
  \item 
  Addition: $(a, b)+(c, d)=(0, b+d)$\\ Scalar Multiplication: $k(a, b)=(ka, kb)$
  \item 
  Addition: $(a, b)+(c, d)=(a+c, b+d)$\\ Scalar Multiplication: $k(a, b)=(a, kb)$
  \item 
  Addition: $(a, b)+(c, d)=(a-c, b-d)$\\ Scalar Multiplication: $k(a, b)=(ka, kb)$
  \end{enumerate}
   
 \item 
 Let $\mathcal{F}$ be the set of all real-valued functions whose domain is all real numbers.  Define addition and scalar multiplication as follows:
 $$(f+g)(x)=f(x)+g(x)\quad (cf)(x)=cf(x)$$
 Verify that $\mathcal{F}$ is a vector space.
 \item
 A differential equation is an equation that contains derivatives.  Consider the differential equation:
 \begin{align}\label{diffeq} f''+f=0\end{align}
A solution to such an equation is a function.
  \begin{enumerate}
  \item Verify that $f(x)=\sin x$ is a solution to (\ref{diffeq}).
  \item Is $f(x)=2\sin x$ a solution?
  \item Is $f(x)=\cos x$ a solution?
  \item Is $f(x)=\sin x+\cos x$ a solution?
  \item Let $S$ be the set of all solutions to (\ref{diffeq}).  Prove that $S$ is a vector space.
  \end{enumerate}
\item
In this problem we will check that the set $\mathbb{C}$ of all complex numbers is in fact a vector space.  Let $z_1 = a_1 + b_1 i$ be a complex number where $i=\sqrt{-1}$.  Similarly, let $z_2 = a_2 + b_2 i$ and $z_3 = a_3 + b_3 i$ be complex numbers and let $k$ and $p$ be scalars.  Check that complex numbers are closed under addition and multiplication, and that they satisfy each of the vector space properties.
\item
Let $S$ be the set of all complex numbers that lie on the unit circle, with usual addition and real-number scalar multiplication. 
%  \begin{enumerate}
 % \item Is $S$ closed under multiplication?
  %\item Is $S$ closed under addition?
  %\item Is $S$ closed under real number scalar multiplication?
  %\end{enumerate}
  Which vector space properties does $S$ satisfy?  Which vector space properties does $S$ fail to satisfy?  Justify your claims.  
  

	
%  \item 
%  Let $Q_1^3$ be the set of all vectors in $\mathbb{R}^2$ located in the first or the third quadrant, or along the axes.
%  \begin{enumerate}
%  \item Is $Q_1^3$ closed under vector addition?
%  \item Is $Q_1^3$ closed under scalar multiplication?
%    \end{enumerate}    

\end{enumerate}
 

\end{document} 